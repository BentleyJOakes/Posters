\documentclass[final,hyperref={pdfpagelabels=false}]{beamer}
\usepackage{grffile}
\mode<presentation>
  {
  %  \usetheme{Berlin}
  \usetheme{I6pd2}
  }
  \usepackage{times}
  \usepackage{amsmath,amsthm, amssymb, latexsym}
  \boldmath
  \usepackage[english]{babel}
  \usepackage[latin1]{inputenc}
%  \usepackage[orientation=portrait,size=a0,scale=1.4,debug]{beamerposter}
  \usepackage[orientation=portrait,size=custom,width=80,height=120,scale=1.4]{beamerposter}  
  
  % my packages
  \usepackage{multirow}

  %%%%%%%%%%%%%%%%%%%%%%%%%%%%%%%%%%%%%%%%%%%%%%%%%%%%%%%%%%%%%%%%%%%%%%%%%%%%%%%%%5
  \graphicspath{{figures/}}
  \title[Fancy Posters]{DSLTrans - Building Correct Model Transformations}
  
  \author[Dreuw \& Deselaers]{Levi L\'ucio$^{\dagger\ddagger}$ and Bruno
  Barroca$^{\ddagger}$}
  
  \institute[McGill University]{$^{\dagger}$McGill University, Montreal,
  Canada\qquad $^{\ddagger}$Universidade Nova de Lisboa, Lisbon, Portugal}
  
  \date[June 14th, 2011]{June 14th, 2011}


  %%%%%%%%%%%%%%%%%%%%%%%%%%%%%%%%%%%%%%%%%%%%%%%%%%%%%%%%%%%%%%%%%%%%%%%%%%%%%%%%%5
  \begin{document}
  \begin{frame}{} 
      \begin{columns}[t]
        \begin{column}{.993\linewidth}
        \begin{block}{\large Problem Statement}
   		Model transformation languages today are mostly general purpose. This
   		bring too much expressiveness to domains which require specific kinds of
   		transformations -- which is a source of bugs and unwanted problems. From
   		the construction of our DSLTrans translation~\cite{SLE2010} language we
   		propose extrapolating a theory for building
   		domain specific transformation languages allowing expressing
   		\emph{terminating} and \emph{confluent} transformations. Additionally,
   		our technique allows that particular transformations are additionally
   		verified by model checking techniques~\cite{MODELS2010}.
        \end{block}
        \end{column}
      \end{columns}         

    \vspace{0.5cm}

    \begin{columns}[t]
      \begin{column}{.485\linewidth}
        \begin{block}{Turing Incompleteness}
        Nearly all transformation languages today are \emph{Turing Complete}.
        This means they have the computational power of a Turing Machine, with also
        its main dropback - termination is undecidable.
        
        DSLTrans is \emph{Turing Incomplete}. This means all transformations
        made in DSLTrans are always guaranteed to terminate. This guarantee
        allows saving previous development time by cutting down on the amount of
        debugging needed.
        \end{block}
      \end{column}

      \begin{column}{.485\linewidth}
        \begin{block}{Correctness-by-Construction}
        We believe in \emph{Correctness-by-Construction}. This means the maximum
        amount of relevant domain specific properties should be guaranteed
        mathematically by the transformation language -- while being transparent
        to the developer.
        
        This is the domain specific approach -- the language allows you and
        helps you to do what you want to do, no more, no less.\vspace{1.45cm}
        \end{block}
      \end{column}

    \end{columns}

    \vspace{0.5cm}
    
    \begin{columns}[t]

      \begin{column}{.485\linewidth}
        \begin{block}{Domain of Language Translation}
		  \includegraphics[width=0.9\linewidth]{figures/meta_model_transf}\\
		  \vspace{.7cm}  
		  {\footnotesize		  
 		  Example translating police station organisation models into officer gender
 		  classification models.}
        \end{block}
      \end{column}

       \begin{column}{.485\linewidth}
        \begin{block}{Visual Language -- Eclipse Based Prototype}
		  \includegraphics[width=0.95\linewidth]{figures/DSLTrans_prototype}
		  \vspace{.01cm}  
		  \begin{itemize}
		  {\footnotesize
		    \item DSLTrans is domain specific to language translation;
		    \item Layered transformations;
		    \item Expressive matching language -- transitive closure, attribute
		    conditions;
		    \item Capability of referring to the results of previous layers;
		    \item Mathematically proved \emph{termination} and
		    \emph{confluence} of DSLTrans transformations.}
		  \end{itemize}
        \end{block}
      \end{column}

    \end{columns}

   \vspace{0.5cm}

   \begin{columns}[t]  
   \begin{column}{.993\linewidth}
   \begin{block}{DSLTrans Transformation Model Checking}
   \begin{centering}
   \begin{tabular}{c c c}
	   \multirow{1}{*}{}
	   		\includegraphics[width=0.25\linewidth]{figures/satisfied} &
	   		\hspace{3cm} \includegraphics[width=0.25\linewidth]{figures/unsatisfied}
	   		\hspace{5cm} & \includegraphics[width=0.25\linewidth]{figures/unProvable} \\  
       \multirow{3}{*}{}
       {\footnotesize ``Any police station that has both a
       female and male chief} & 
       {\footnotesize ``Any female officer in the source model will
       always} &
       {\footnotesize ``Any female officer in the source
       model supervising}
        \\
       
       {\footnotesize  officers will include those officers in the
       female and male} & 
       {\footnotesize supervise a male officer in the target model.''} &
       {\footnotesize ``a male officer will keep on supervising the same}
        \\
       
       {\footnotesize lists in the target model.''} & &
       {\footnotesize  male officer in the target model.''}
        \\
       {\footnotesize \textbf{TRUE} for all transformation instances.} &
       {\footnotesize \textbf{FALSE} for all transformation instances.} &
       {\footnotesize \textbf{NON PROVABLE}.}
        \\

 	\end{tabular}
   \end{centering} 	
   \end{block}
   \end{column}
   \end{columns}
  
   \vspace{0.5cm}  
  
   \begin{columns}[t] 
    \begin{column}{.485\linewidth}
    	\begin{block}{Conclusions \& Future Work}
   	   The DSLTrans principles can be extended to building other domain
   	   specific translation languages. We are currently thinking about building domain
       specific translation languages by composing components such that the
       resulting language will always describe \emph{finite} and
       \emph{confluent} transformations. If that is the case, transformation
       model checkers such as the one for DSLTrans can also be
       (semi-)automatically built.\vspace{1.1cm}
        \end{block}
        
    \end{column}
    \begin{column}{.485\linewidth}
    	\begin{block}{Bibliography}
    	   \begin{thebibliography}{10} 
    	   \bibitem{SLE2010} Bruno Barroca, Levi Lucio, Vasco Amaral, Roberto
    	   F{\'e}lix and Vasco Sousa, {\em DSLTrans: A Turing Incomplete
    	   Transformation Language}, Proceedings of the SLE 2010 conference,
    	   Springer 2010, pp. 296-305.
    	   \bibitem{MODELS2010} Levi Lucio, Bruno Barroca, Vasco Amaral, {\em A
    	   Technique for Automatic Validation of Model Transformations},
    	   Proceedings of the MoDELS 2010 Conference, Springer, pp. 136-150.
		\end{thebibliography}	   
    	\end{block}        
    \end{column}    
   \end{columns}

  \end{frame}
\end{document}